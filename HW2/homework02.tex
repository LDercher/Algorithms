\documentclass{exam} % {{{1
  \usepackage{amsmath, amssymb, amsthm, enumitem, float, caption, mathtools, tikz}
  \usetikzlibrary{arrows, calc, decorations.markings, matrix, positioning}
  \tikzset{>=latex}
  \usepackage[draft]{hyperref}
  
  % mathbb and mathcal symbols
  \newcommand{\NN}{\mathbb{N}}
  \newcommand{\ZZ}{\mathbb{Z}}
  \newcommand{\QQ}{\mathbb{Q}}
  \newcommand{\V}{\mathcal{V}}
  \newcommand{\A}{\mathbb{A}}
  \newcommand{\m}[1]{\mathbb{#1}}    % for models
  \newcommand{\cl}[1]{\mathcal{#1}}  % for classes
  
  % theorems and similar environments
  \theoremstyle{plain} 
    \newtheorem{thm}{Theorem}[section]  \newtheorem*{thm*}{Theorem}
    \newtheorem{claim}{Claim}[thm]      \newtheorem*{claim*}{Claim}
    \newtheorem{conj}{Conjecture}[thm]  \newtheorem*{conj*}{Conjecture}
    \newtheorem{cor}{Corollary}[thm]    \newtheorem*{cor*}{Corollary}
    \newtheorem{lem}{Lemma}[thm]        \newtheorem*{lem*}{Lemma}
    \newtheorem{prop}{Proposition}[thm] \newtheorem*{prop*}{Proposition}
  \theoremstyle{definition}
    \newtheorem{defn}{Definition}[thm] \newtheorem*{defn*}{Definition}
  \theoremstyle{remark}
    \newtheorem{rk}{Remark}[thm]  \newtheorem*{rk*}{Remark}
    \newtheorem{ex}{Example}[thm] \newtheorem*{ex*}{Example}
  \newcommand{\Case}[1]{\smallskip \textbf{Case #1:}}
  \newenvironment{claimproof} {
    \begin{proof}[Proof of claim]
    \renewcommand{\qedsymbol}{\ensuremath{\bullet}}
    } {
    \end{proof}
    }
  
  % custom commands
  \DeclareMathOperator{\Cg}{Cg}
  \DeclareMathOperator{\Clo}{Clo}
  \DeclareMathOperator{\Con}{Con}
  \DeclareMathOperator{\Rel}{Rel}
  \DeclareMathOperator{\Sg}{Sg}
  \DeclareMathOperator{\diag}{diag}
  \newcommand{\bmat}[1]{\ensuremath{ \begin{bmatrix} #1 \end{bmatrix} }}
  \newcommand{\Bmat}[1]{\ensuremath{ \begin{Bmatrix} #1 \end{Bmatrix} }}
  \newcommand{\pmat}[1]{\ensuremath{ \begin{pmatrix} #1 \end{pmatrix} }}
  \newcommand{\ds}[1]{\ensuremath{ \displaystyle{#1} }}
  \newcommand{\stack}[2]{\genfrac{}{}{0pt}{}{#1}{#2}}
  
  % misc
  \pagestyle{foot} \cfoot{Page \thepage\ of \numpages}  % page numbering
  \numberwithin{equation}{section}  % number equations within sections
  \renewcommand{\d}{\;d}
  \renewcommand{\epsilon}{\varepsilon}
  \renewcommand{\phi}{\varphi}
  
  % exam documentclass settings
  % restyle parts and subparts
  \renewcommand{\thepartno}{\roman{partno}}
  \renewcommand{\thesubpart}{\alph{subpart}}
  \renewcommand{\subpartlabel}{(\thesubpart)}
  \renewcommand{\subsubpartlabel}{(\thesubsubpart)}
  % restyle multiple choice options
  \renewcommand{\choicelabel}{\thechoice)}
  % true or false questions (use \TFQuestion)
  \newcommand{\TrueFalse}{\hspace*{0.25em}\textbf{True}\hspace*{1.25em}\textbf{False}\hspace*{1em}}
  \newlength{\mylena} \newlength{\mylenb} \settowidth{\mylena}{\TrueFalse}
  \newcommand{\TFQuestion}[1]{
    \setlength{\mylenb}{\linewidth} 
    \addtolength{\mylenb}{-121.15pt}
    \parbox[t]{\mylena}{\TrueFalse}\parbox[t]{\mylenb}{#1}
  }
  
  % document specific stuff
  \usepackage[final]{minted}
  \usemintedstyle{pastie}
  \renewcommand{\O}{\mathcal{O}}
  %----------------------------------------------------------------------------}}}1
  
  \begin{document}  % \printanswers
  % title header {{{
  \title{Fundamentals of Computer Algorithms \\ Homework 2 Additional Problems}
  \author{Prof. Matthew Moore}
  \date{Due: 2018-09-04}
  \maketitle
  %----------------------------------------------------------------------------}}}
  \begin{questions}
  % heaps, python   {{{1
  % ID: eagais2V
  \question In the python file associated with this homework, implement the
  functions
  \begin{enumerate}
    \item \verb|Heap._parent_index|,
    \item \verb|Heap._parent|,
    \item \verb|Heap._parent_key|,
    \item \verb|Heap._heapify_up|.
  \end{enumerate}
  Follow the instructions in the source file. In your printed homework
  submission, include \emph{only} these functions, do not print the entire
  source file.
  %----------------------------------------------------------------------------}}}1
  % priority queue, python   {{{1
  % ID: Aige0aiM
  \question In the python file associated with this homework, implement the
  functions
  \begin{enumerate}
    \item \texttt{PQ.add},
    \item \texttt{PQ.pop}.
  \end{enumerate}
  Follow the instructions in the source file. In your printed homework
  submission, include \emph{only} these functions, do not print the entire
  source file.
  %----------------------------------------------------------------------------}}}1
  % heapsort, python   {{{1
  % ID: fooH4vee
  \question In the python file associated with this homework, implement the
  function \verb|sort_with_PQ|.
  
  \smallskip
  
  Your solution should be linear in the priority queue operations. That is, it
  should run in $\O(n\cdot\O(PQ))$, where $\O(PQ)$ is the $\O$-complexity of
  the priority queue operations. Since the heap operations are all
  $\O(\log(n))$, this sort algorithm is $\O(n\log(n))$. It is called
  \emph{heapsort}, and is used by default in the Linux kernel due to its more
  stable performance characteristics.
  %----------------------------------------------------------------------------}}}1
  % algorithm refinement   {{{1
  % ID: Vaiz5moo
  \question Let $L$ be a list and define property (A) on elements of $L$ by
  \[ \tag{A}
    \text{$L[i]$ is not smaller than its neighbors, if they exist}.
  \]
  ``Neighbors'' refers to the elements $L[i-1]$ and $L[i+1]$ if they exist.
  Here is an algorithm that will find \emph{one} element of $L$ exhibiting
  property (A).
  \begin{minted}{python}
    def findA(L):
      for i in range(len(L)):
        left = i-1
        right = i+1
        if i == 0:
          left = i
        if i == len(L)-1:
          right = i
        if L[left] <= L[i] >= L[right]:
          return L[i]
  \end{minted}
  \begin{parts}
    \part What is the $\Theta$-complexity of \verb|findA|?
  
    \part Implement a function \verb|better_findA| that performs better than
      \verb|findA|. That is, if $f(n)$ and $b(n)$ are the number of iterations
      of \verb|findA| and \verb|better_findA|, respectively, then
      \[
        \lim_{n\to\infty} \frac{b(n)}{f(n)} = 0.
      \]
  \end{parts}
  %----------------------------------------------------------------------------}}}1
  % complexity {{{1
  % ID: EeFiba9O
  \question Suppose we have two parameters, $m$ and $n$, with $m\to \infty$
  and $n\to \infty$, perhaps at different rates independent of one another.
  Which has larger $\Theta$-complexity: $m^{\ln(n)}$ or $n^{\ln(m)}$?
  %----------------------------------------------------------------------------}}}1
  \end{questions} \end{document}
  