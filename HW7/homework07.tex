\documentclass{exam} % {{{1
\usepackage{amsmath, amssymb, amsthm, enumitem, float, caption, mathtools, tikz}
\usetikzlibrary{arrows, calc, decorations.markings, matrix, positioning}
\tikzset{>=latex}
\usepackage[final]{hyperref}

% mathbb and mathcal symbols
\newcommand{\NN}{\mathbb{N}}
\newcommand{\ZZ}{\mathbb{Z}}
\newcommand{\QQ}{\mathbb{Q}}
\newcommand{\V}{\mathcal{V}}
\newcommand{\A}{\mathbb{A}}
\newcommand{\m}[1]{\mathbb{#1}}    % for models
\newcommand{\cl}[1]{\mathcal{#1}}  % for classes

% theorems and similar environments
\theoremstyle{plain} 
  \newtheorem{thm}{Theorem}[section]  \newtheorem*{thm*}{Theorem}
  \newtheorem{claim}{Claim}[thm]      \newtheorem*{claim*}{Claim}
  \newtheorem{conj}{Conjecture}[thm]  \newtheorem*{conj*}{Conjecture}
  \newtheorem{cor}{Corollary}[thm]    \newtheorem*{cor*}{Corollary}
  \newtheorem{lem}{Lemma}[thm]        \newtheorem*{lem*}{Lemma}
  \newtheorem{prop}{Proposition}[thm] \newtheorem*{prop*}{Proposition}
\theoremstyle{definition}
  \newtheorem{defn}{Definition}[thm] \newtheorem*{defn*}{Definition}
\theoremstyle{remark}
  \newtheorem{rk}{Remark}[thm]  \newtheorem*{rk*}{Remark}
  \newtheorem{ex}{Example}[thm] \newtheorem*{ex*}{Example}
\newcommand{\Case}[1]{\smallskip \textbf{Case #1:}}
\newenvironment{claimproof} {
  \begin{proof}[Proof of claim]
  \renewcommand{\qedsymbol}{\ensuremath{\bullet}}
  } {
  \end{proof}
  }

% custom commands
\DeclareMathOperator{\Cg}{Cg}
\DeclareMathOperator{\Clo}{Clo}
\DeclareMathOperator{\Con}{Con}
\DeclareMathOperator{\Rel}{Rel}
\DeclareMathOperator{\Sg}{Sg}
\DeclareMathOperator{\diag}{diag}
\newcommand{\bmat}[1]{\ensuremath{ \begin{bmatrix} #1 \end{bmatrix} }}
\newcommand{\Bmat}[1]{\ensuremath{ \begin{Bmatrix} #1 \end{Bmatrix} }}
\newcommand{\pmat}[1]{\ensuremath{ \begin{pmatrix} #1 \end{pmatrix} }}
\newcommand{\mat}[1]{\ensuremath{ \begin{matrix} #1 \end{matrix} }}
\newcommand{\vect}[1]{\left< #1 \right>}
\newcommand{\ds}[1]{\ensuremath{ \displaystyle{#1} }}
\newcommand{\stack}[2]{\genfrac{}{}{0pt}{}{#1}{#2}}

% misc
\pagestyle{foot} \cfoot{Page \thepage\ of \numpages}  % page numbering
\numberwithin{equation}{section}  % number equations within sections
\renewcommand{\d}{\;d}
\renewcommand{\epsilon}{\varepsilon}
\renewcommand{\phi}{\varphi}

% exam documentclass settings
% restyle parts and subparts
\renewcommand{\thepartno}{\roman{partno}}
\renewcommand{\thesubpart}{\alph{subpart}}
\renewcommand{\subpartlabel}{(\thesubpart)}
\renewcommand{\subsubpartlabel}{(\thesubsubpart)}
% restyle multiple choice options
\renewcommand{\choicelabel}{\thechoice)}
% true or false questions (use \TFQuestion)
\newcommand{\TrueFalse}{\hspace*{0.25em}\textbf{True}\hspace*{1.25em}\textbf{False}\hspace*{1em}}
\newlength{\mylena} \newlength{\mylenb} \settowidth{\mylena}{\TrueFalse}
\newcommand{\TFQuestion}[1]{
  \setlength{\mylenb}{\linewidth} 
  \addtolength{\mylenb}{-121.15pt}
  \parbox[t]{\mylena}{\TrueFalse}\parbox[t]{\mylenb}{#1}
}

% document specific stuff
\usepackage[final]{minted}
\usemintedstyle{pastie}
\renewcommand{\O}{\mathcal{O}}
%----------------------------------------------------------------------------}}}1

\begin{document}  % \printanswers
% title header {{{
\title{Fundamentals of Computer Algorithms \\ Homework 7 Additional Problems}
\author{Prof. Matthew Moore}
\date{Due: 2018-10-09}
\maketitle
%----------------------------------------------------------------------------}}}
\begin{questions}
% graphs   {{{1
% ID: Quietoo8
\question Let $\m{G}$ be a directed graph. Prove or disprove: if every node
$n\in G$ has $\text{in-deg}(n) > 0$, then $\m{G}$ has a cycle.
%----------------------------------------------------------------------------}}}1

% minimum spanning trees   {{{1
% ID: hoon5Yuv
\question Let $\m{G}$ be a graph. We say an edge $(u,v)$ \emph{crosses} the
cut $S$ if it has one edge in $S$ and the other in $G\setminus S$. We say
that an edge $(u,v)$ is \emph{light} for $S$ if it has minimum weight
amongst all the edges which cross $S$.

\medskip

Let
\[
  M
  = \Big\{ \text{edges } (u,v) \mid \text{for some cut $S$, $(u,v)$ is light} \Big\}.
\]
Give a simple example of a connected graph $\m{G}$ such that $M$ does not
form the edges of a minimum spanning tree.
%----------------------------------------------------------------------------}}}1

% minimum spanning trees   {{{1
% ID: oogeCh6a
\question Suppose that all the edge weights of $\m{G}$ are positive and the
$\m{G}$ does not have any self-loops.
\begin{parts}
  \part Prove that any subset of the edges which connects all vertices and
    has minimum total weight must be a tree.
  \part If we allow edge weights to also be negative, is the previous part
    still true? If it is, then prove it. If it isn't then provide a
    counterexample.
\end{parts}
%----------------------------------------------------------------------------}}}1
\end{questions} \end{document}
