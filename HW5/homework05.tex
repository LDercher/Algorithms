\documentclass{exam} % {{{1
\usepackage{amsmath, amssymb, amsthm, enumitem, float, caption, mathtools, tikz}
\usetikzlibrary{arrows, calc, decorations.markings, matrix, positioning}
\tikzset{>=latex}
\usepackage[final]{hyperref}

% mathbb and mathcal symbols
\newcommand{\NN}{\mathbb{N}}
\newcommand{\ZZ}{\mathbb{Z}}
\newcommand{\QQ}{\mathbb{Q}}
\newcommand{\V}{\mathcal{V}}
\newcommand{\A}{\mathbb{A}}
\newcommand{\m}[1]{\mathbb{#1}}    % for models
\newcommand{\cl}[1]{\mathcal{#1}}  % for classes

% theorems and similar environments
\theoremstyle{plain} 
  \newtheorem{thm}{Theorem}[section]  \newtheorem*{thm*}{Theorem}
  \newtheorem{claim}{Claim}[thm]      \newtheorem*{claim*}{Claim}
  \newtheorem{conj}{Conjecture}[thm]  \newtheorem*{conj*}{Conjecture}
  \newtheorem{cor}{Corollary}[thm]    \newtheorem*{cor*}{Corollary}
  \newtheorem{lem}{Lemma}[thm]        \newtheorem*{lem*}{Lemma}
  \newtheorem{prop}{Proposition}[thm] \newtheorem*{prop*}{Proposition}
\theoremstyle{definition}
  \newtheorem{defn}{Definition}[thm] \newtheorem*{defn*}{Definition}
\theoremstyle{remark}
  \newtheorem{rk}{Remark}[thm]  \newtheorem*{rk*}{Remark}
  \newtheorem{ex}{Example}[thm] \newtheorem*{ex*}{Example}
\newcommand{\Case}[1]{\smallskip \textbf{Case #1:}}
\newenvironment{claimproof} {
  \begin{proof}[Proof of claim]
  \renewcommand{\qedsymbol}{\ensuremath{\bullet}}
  } {
  \end{proof}
  }

% custom commands
\DeclareMathOperator{\Cg}{Cg}
\DeclareMathOperator{\Clo}{Clo}
\DeclareMathOperator{\Con}{Con}
\DeclareMathOperator{\Rel}{Rel}
\DeclareMathOperator{\Sg}{Sg}
\DeclareMathOperator{\diag}{diag}
\newcommand{\bmat}[1]{\ensuremath{ \begin{bmatrix} #1 \end{bmatrix} }}
\newcommand{\Bmat}[1]{\ensuremath{ \begin{Bmatrix} #1 \end{Bmatrix} }}
\newcommand{\pmat}[1]{\ensuremath{ \begin{pmatrix} #1 \end{pmatrix} }}
\newcommand{\ds}[1]{\ensuremath{ \displaystyle{#1} }}
\newcommand{\stack}[2]{\genfrac{}{}{0pt}{}{#1}{#2}}

% misc
\pagestyle{foot} \cfoot{Page \thepage\ of \numpages}  % page numbering
\numberwithin{equation}{section}  % number equations within sections
\renewcommand{\d}{\;d}
\renewcommand{\epsilon}{\varepsilon}
\renewcommand{\phi}{\varphi}

% exam documentclass settings
% restyle parts and subparts
\renewcommand{\thepartno}{\roman{partno}}
\renewcommand{\thesubpart}{\alph{subpart}}
\renewcommand{\subpartlabel}{(\thesubpart)}
\renewcommand{\subsubpartlabel}{(\thesubsubpart)}
% restyle multiple choice options
\renewcommand{\choicelabel}{\thechoice)}
% true or false questions (use \TFQuestion)
\newcommand{\TrueFalse}{\hspace*{0.25em}\textbf{True}\hspace*{1.25em}\textbf{False}\hspace*{1em}}
\newlength{\mylena} \newlength{\mylenb} \settowidth{\mylena}{\TrueFalse}
\newcommand{\TFQuestion}[1]{
  \setlength{\mylenb}{\linewidth} 
  \addtolength{\mylenb}{-121.15pt}
  \parbox[t]{\mylena}{\TrueFalse}\parbox[t]{\mylenb}{#1}
}

% document specific stuff
\usepackage[final]{minted}
\usemintedstyle{pastie}
\renewcommand{\O}{\mathcal{O}}
%----------------------------------------------------------------------------}}}1

\begin{document}  % \printanswers
% title header {{{
\title{Fundamentals of Computer Algorithms \\ Homework 5 Additional Problems}
\author{Prof. Matthew Moore}
\date{Due: 2018-09-25}
\maketitle
%----------------------------------------------------------------------------}}}
\begin{questions}
% topological sort   {{{1
% ID: shi8uYie
\question \begin{parts}
  \part Implement the function \verb|topological_sort| in the attached
    python file. Your algorithm should be of \emph{linear} complexity in the
    vertices and edges of the graph. Submit only your function in your
    printed homework.
  \part Use \verb|randgraph_DAG| to generate a random graph with 10
    vertices. (This can take a \emph{long} time.) Draw this graph.
  \part Run your \verb|topological_sort| function on your graph from the
    previous part. Draw the resulting topologically sorted graph.
\end{parts}
%----------------------------------------------------------------------------}}}1
% DAGs   {{{1
% ID: Ichoo9aB
\question \begin{parts}
  \part Implement the function \verb|is_DAG| in the attached python file.
    Your algorithm should be of linear complexity in the vertices and edges
    of the graph. Submit only your function in your printed homework.
  \part Explain why your algorithm works.
\end{parts}
%----------------------------------------------------------------------------}}}1
% Hamiltonian paths   {{{1
% ID: ees5iiPh
\question \begin{parts}
  \part Implement the function \verb|findHamiltonian_DAG| in the attached
    python file. Your algorithm should be of \emph{linear} complexity in the
    vertices and edges of the graph. Submit only your function in your
    printed homework.
  \part Use \verb|randgraph_DAGwithHam| to generate a random graph with 10
    vertices. (This can take a \emph{very} long time.) Draw this graph.
  \part Run your \verb|findHamiltonian_DAG| function on the graph from the
    previous part and indicate the Hamiltonian path in your drawing.
  \part Run your \verb|topological_sort| function on your graph from the
    previous part. Draw the resulting topologically sorted graph.
\end{parts}
%----------------------------------------------------------------------------}}}1
% interval scheduling   {{{1
% ID: iepo6Ca5
\question In the interval scheduling problem we always select the interval
which finishes first and which is compatible with all previous selections.
Suppose instead that we select the interval which \emph{starts last} and
which is compatible with all previous selections.
\begin{parts}
  \part Write a pseudocode algorithm with complexity $\O(n\log n)$
    implementing this strategy.
  \part Does this algorithm always return an optimal solution? If it
    doesn't, then provide a counterexample. If it does, then prove it.
\end{parts}
%----------------------------------------------------------------------------}}}1
% interval scheduling   {{{1
% ID: Pai6aeXa
\question For each of the parts below, show by producing a counterexample
that the method of selecting intervals does not produce an optimal solution
to the interval scheduling problem.
\begin{parts}
  \part Select intervals with earliest starting time and compatible with all
    previous selections.
  \part Select intervals with the shortest duration and compatible with all
    previous selections.
\end{parts}
%----------------------------------------------------------------------------}}}1
\end{questions} \end{document}
